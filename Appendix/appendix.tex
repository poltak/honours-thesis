\chapter{Samza Pipeline Prototype Implementation}
\label{lst:samza}
\begin{lstlisting}[language=java,caption=edu.monash.honours.system.SpeedConsumer (Java)]
package edu.monash.honours.system;

import edu.monash.honours.util.SpeedReadingPair;
import org.apache.samza.Partition;
import org.apache.samza.system.IncomingMessageEnvelope;
import org.apache.samza.system.SystemStreamPartition;
import org.apache.samza.util.BlockingEnvelopeMap;

import java.io.BufferedReader;
import java.io.IOException;
import java.io.InputStream;
import java.io.InputStreamReader;
import java.net.ServerSocket;

public class SpeedConsumer extends BlockingEnvelopeMap
{
  private final ServerSocket serverSocket;
  private final SystemStreamPartition systemStreamPartition;

  public SpeedConsumer(final String systemName, final String streamName, final int listeningPort) throws IOException
  {
    this.systemStreamPartition = new SystemStreamPartition(systemName, streamName, new Partition((0)));
    this.serverSocket = new ServerSocket(listeningPort);
  }

  @Override
  public void start()
  {
    new Thread(
      () -> {
        try {
          SpeedConsumer.this.listenOnPort();
        } catch (InterruptedException e) {
          SpeedConsumer.this.stop();
        }
      }
    );
  }

  private void listenOnPort() throws InterruptedException
  {
    InputStream inputStream;

    try {
      while ((inputStream = serverSocket.accept().getInputStream()) != null) {
        BufferedReader bufferedReader = new BufferedReader(new InputStreamReader(inputStream));
        double speedReading = Double.valueOf(bufferedReader.readLine());

        put(systemStreamPartition,
            new IncomingMessageEnvelope(systemStreamPartition, null, null, new SpeedReadingPair(speedReading, false)));

        bufferedReader.close();
      }
    } catch (IOException e) {
      put(systemStreamPartition, new IncomingMessageEnvelope(systemStreamPartition, null, null,
                                                             "ERROR: Cannot read from port" + e.getMessage()));
    }
  }







  @Override
  public void stop()
  {
    try {
      this.serverSocket.close();
    } catch (IOException e) {
      e.printStackTrace();
    }
  }

  @Override
  public void register(final SystemStreamPartition systemStreamPartition, final String startingOffset)
  {
    super.register(systemStreamPartition, startingOffset);
  }
}
\end{lstlisting}

\clearpage


\begin{lstlisting}[language=java,caption=edu.monash.honours.system.SpeedSystemFactory (Java)]
package edu.monash.honours.system;

import org.apache.samza.SamzaException;
import org.apache.samza.config.Config;
import org.apache.samza.metrics.MetricsRegistry;
import org.apache.samza.system.SystemAdmin;
import org.apache.samza.system.SystemConsumer;
import org.apache.samza.system.SystemFactory;
import org.apache.samza.system.SystemProducer;
import org.apache.samza.util.SinglePartitionWithoutOffsetsSystemAdmin;

import java.io.IOException;

public class SpeedSystemFactory implements SystemFactory
{
  private static final String OUTPUT_STREAM_NAME = "speeds";

  @Override
  public SystemConsumer getConsumer(final String systemName, final Config config, final MetricsRegistry metricsRegistry)
  {
    int listeningPort = Integer.getInteger(config.get("systems." + systemName + ".listeningPort"));

    try {
      return new SpeedConsumer(systemName, OUTPUT_STREAM_NAME, listeningPort);
    } catch (IOException e) {
      e.printStackTrace();
      return null;
    }
  }

  @Override
  public SystemProducer getProducer(final String s, final Config config, final MetricsRegistry metricsRegistry)
  {
    throw new SamzaException("You cannot produce to this feed!");
  }






  @Override
  public SystemAdmin getAdmin(final String s, final Config config)
  {
    return new SinglePartitionWithoutOffsetsSystemAdmin();
  }
}
\end{lstlisting}

\clearpage
\begin{lstlisting}[language=java,caption=edu.monash.honours.task.SpeedCheckStreamTask (Java)]
package edu.monash.honours.task;

import edu.monash.honours.util.SpeedReadingPair;
import org.apache.samza.system.IncomingMessageEnvelope;
import org.apache.samza.system.OutgoingMessageEnvelope;
import org.apache.samza.system.SystemStream;
import org.apache.samza.task.MessageCollector;
import org.apache.samza.task.StreamTask;
import org.apache.samza.task.TaskCoordinator;

import java.util.Map;

public class SpeedCheckStreamTask implements StreamTask
{
  // Constants
  private static final SystemStream FLAGGED_OUTPUT     = new SystemStream("kafka", "flagged-output");
  private static final SystemStream SPEED_CHECK_OUTPUT = new SystemStream("kafka", "speed-check-output");
  private static final double       SPEED_UPPER_LIMIT  = 90;
  private static final double       SPEED_LOWER_LIMIT  = 0;

  private static int totalMessageCount = 0;
  private int messageCount;

  public SpeedCheckStreamTask()
  {
    this.messageCount = 0;
  }















  /**
   * Listens on a stream providing speed readings, and processes those speed readings depending on whether or not they
   * exceed a specified threshold.
   *
   * Forwards on speed value to flagged stream if that value exceeds thresholds, else forwards on speed for further
   * processing.
   */
  public void process(final IncomingMessageEnvelope incomingMessageEnvelope, final MessageCollector messageCollector,
                      final TaskCoordinator taskCoordinator) throws Exception
  {
    SpeedReadingPair speedReadingPair = (SpeedReadingPair) incomingMessageEnvelope.getMessage();
    this.messageCount++;  // Keep count of number of messages that come through here
    totalMessageCount += this.messageCount; // Keep count of total number of messages received on this stream.


    if (speedReadingPair.getSpeed() < SPEED_LOWER_LIMIT || speedReadingPair.getSpeed() > SPEED_UPPER_LIMIT) {
      // Flag speeds that exceed thresholds
      speedReadingPair.flag();
      messageCollector.send(new OutgoingMessageEnvelope(FLAGGED_OUTPUT, speedReadingPair));
    }

    // Send speed on for further processing
    messageCollector.send(new OutgoingMessageEnvelope(SPEED_CHECK_OUTPUT, speedReadingPair));
  }
}
\end{lstlisting}

\clearpage
\begin{lstlisting}[language=java,caption=edu.monash.honours.util.SpeedReadingPair (Java)]
package edu.monash.honours.util;

import java.util.AbstractMap;
import java.util.HashMap;
import java.util.Map;

public class SpeedReadingPair
{
  private final double key;
  private boolean value;

  public SpeedReadingPair(final double key, final boolean value)
  {
    this.key = key;
    this.value = value;
  }

  public SpeedReadingPair(Map<String, Object> jsonObject)
  {
    this((Double) jsonObject.get("speed"), (Boolean) jsonObject.get("flag"));
  }

  @Override
  public String toString()
  {
    return "{" + this.key + ":" + this.value + "}";
  }

  public double getSpeed()
  {
    return this.key;
  }

  public boolean flagged()
  {
    return this.value;
  }

  public void flag()
  {
    this.value = true;
  }

  public static Map<String, Object> toMap(SpeedReadingPair pair)
  {
    Map<String, Object> jsonObject = new HashMap<String, Object>();

    jsonObject.put("speed", pair.getSpeed());
    jsonObject.put("flag", pair.flagged());

    return jsonObject;
  }

  public static String toJson(SpeedReadingPair pair)
  {
    return pair.toString();
  }
}
\end{lstlisting}

\clearpage
\begin{lstlisting}[caption=SpeedCheckStreamTask (Samza config)]
# Job
job.factory.class=org.apache.samza.job.yarn.YarnJobFactory
job.name=speed-check


# YARN
yarn.package.path=file://${basedir}/target/${project.artifactId}-${pom.version}-dist.tar.gz


# Task
task.class=edu.monash.honours.task.SpeedCheckStreamTask
task.inputs=speed-input


# Serializers
serializers.registry.json.class=org.apache.samza.serializers.JsonSerdeFactory


# speed-input System
systems.speed-input.samza.factory=edu.monash.honours.system.SpeedSystemFactory
systems.speed-input.listeningPort=9999


# Kafka System
systems.kafka.samza.factory=org.apache.samza.system.kafka.KafkaSystemFactory
systems.kafka.samza.msg.serde=json
systems.kafka.consumer.zookeeper.connect=localhost:2181/
systems.kafka.producer.bootstrap.servers=localhost:9092
\end{lstlisting}


\clearpage
\chapter{Storm Pipeline Prototype Implementation}
\label{lst:storm}

\begin{lstlisting}[language=java,caption=edu.monash.honours.bolt.SpeedCheckBolt (Java)]
package edu.monash.honours.bolt;

import backtype.storm.task.OutputCollector;
import backtype.storm.task.TopologyContext;
import backtype.storm.topology.OutputFieldsDeclarer;
import backtype.storm.topology.base.BaseRichBolt;
import backtype.storm.tuple.Fields;
import backtype.storm.tuple.Tuple;
import backtype.storm.tuple.Values;

import java.util.Map;

public class SpeedCheckBolt extends BaseRichBolt
{
  private final static int SPEED_UPPER_BOUND = 90;
  private final static int SPEED_LOWER_LIMIT = 0;

  private static int      totalTupleCount = 0;

  private int             tupleCount;
  private OutputCollector collector;






  /**
   * Runs once when this bolt is created.
   */
  public void prepare(final Map map, final TopologyContext topologyContext, final OutputCollector outputCollector)
  {
    this.collector = outputCollector;
    this.tupleCount = 0;
  }

  /**
   * Runs every time a tuple is received.
   *
   * @param tuple The input tuple containing a single speed field.
   */
  public void execute(final Tuple tuple)
  {
    updateTupleCounts();

    Object receivedValue = tuple.getValue(0);

    Values emitTuple;
    if (receivedValue instanceof Double) {
      // If received value is a double, check the speed received
      emitTuple = checkSpeed((Double) receivedValue);
    } else {
      // else, ignore processing and send the message onwards
      emitTuple = new Values(receivedValue, true);
    }

    this.collector.emit(tuple, emitTuple);
    this.collector.ack(tuple);
  }

  /**
   * Declare the output fields as being speed and a noise flag.
   */
  public void declareOutputFields(final OutputFieldsDeclarer outputFieldsDeclarer)
  {
    outputFieldsDeclarer.declare(new Fields("speed", "noiseFlag"));
  }


  /**
   * Keeps count of tuples that pass through this bolt.
   */
  private void updateTupleCounts()
  {
    this.tupleCount++;
    totalTupleCount += this.tupleCount;
  }

  /**
   * Checks the given speed to see if it within limits.
   *
   * @param speed Speed to check.
   * @return A Values tuple with two fields: speed and a noise flag. Noise flag is checked if speed exceeds limits.
   */
  private Values checkSpeed(double speed)
  {
    if (speed > SPEED_UPPER_BOUND || speed < SPEED_LOWER_LIMIT) {
      return new Values(speed, true);
    }

    return new Values(speed, false);
  }
}
\end{lstlisting}

\clearpage
\begin{lstlisting}[language=java,caption=edu.monash.honours.spout.SpeedOutputSpout (Java)]
package edu.monash.honours.spout;

import backtype.storm.spout.SpoutOutputCollector;
import backtype.storm.task.TopologyContext;
import backtype.storm.topology.OutputFieldsDeclarer;
import backtype.storm.topology.base.BaseRichSpout;
import backtype.storm.tuple.Fields;
import backtype.storm.tuple.Values;

import java.io.BufferedReader;
import java.io.IOException;
import java.io.InputStreamReader;
import java.net.ServerSocket;
import java.util.Map;

public class SpeedOutputSpout extends BaseRichSpout
{
  /**
   * The key in the storm config under which the listening port is declared
   */
  private static final String LISTENING_PORT_CONFIG_KEY = "spout.listeningPort";

  private SpoutOutputCollector collector;
  private ServerSocket         serverSocket;

  /**
   * Runs once when this spout is created.
   *
   * @param conf Storm topology configuration map.
   * @param topologyContext
   * @param spoutOutputCollector
   */
  public void open(final Map conf, final TopologyContext topologyContext,
                   final SpoutOutputCollector spoutOutputCollector)
  {
    this.collector = spoutOutputCollector;

    try {
      this.serverSocket = new ServerSocket(Integer.valueOf((String) conf.get(LISTENING_PORT_CONFIG_KEY)));
    } catch (IOException e) {
      this.collector.emit(new Values("ERROR: Cannot open port -\n" + e.getMessage()));
    }
  }

  /**
   * Run every time a new value is sent to this spout.
   *
   * Emits a tuple containing a single speed value for further processing.
   */
  public void nextTuple()
  {
    try {
      BufferedReader bufferedReader = new BufferedReader(
        (new InputStreamReader(serverSocket.accept().getInputStream())));
      double speedReading = Double.valueOf(bufferedReader.readLine());

      this.collector.emit(new Values(speedReading));
    } catch (IOException e) {
      this.collector.emit(new Values("ERROR: Cannot read from port -\n" + e.getMessage()));
    }
  }

  /**
   * Declare the output field as being a single speed value.
   */
  public void declareOutputFields(final OutputFieldsDeclarer outputFieldsDeclarer)
  {
    outputFieldsDeclarer.declare(new Fields("speed"));
  }
}
\end{lstlisting}
\clearpage

\begin{lstlisting}[language=java,caption=edu.monash.honours.SpeedCheckTopology (Java)]
package edu.monash.honours;

import backtype.storm.Config;
import backtype.storm.LocalCluster;
import backtype.storm.topology.TopologyBuilder;
import backtype.storm.utils.Utils;
import edu.monash.honours.bolt.SpeedCheckBolt;
import edu.monash.honours.spout.SpeedOutputSpout;

public class SpeedCheckTopology
{
  /**
   * The listening port upon which the spout will get data.
   */
  public static final String LISTENING_PORT = "8888";

  public static void main(String[] args) throws Exception
  {
    TopologyBuilder builder = new TopologyBuilder();

    builder.setSpout("speed", new SpeedOutputSpout());
    builder.setBolt("speed-check", new SpeedCheckBolt()).shuffleGrouping("speed");

    Config conf = new Config();
    conf.put("spout.listeningPort", LISTENING_PORT);
    conf.setDebug(true);

    LocalCluster cluster = new LocalCluster();
    cluster.submitTopology("speed-checker", conf, builder.createTopology());
    Utils.sleep(100000);
    cluster.killTopology("speed-checker");
    cluster.shutdown();
  }
}
\end{lstlisting}

\clearpage
\chapter{Spark Streaming Pipeline Prototype Implementations}
\label{lst:spark}

\begin{lstlisting}[language=scala,caption=edu.monash.honours.SpeedCheck (Scala)]
package edu.monash.honours

import org.apache.spark.SparkConf
import org.apache.spark.streaming.{StreamingContext, Seconds}

object SpeedCheck
{
  val SPEED_UPPER_LIMIT = 85
  val SPEED_LOWER_LIMIT = 0

  def main(args: Array[String])
  {
    val sparkConf = new SparkConf().setMaster("local[2]").setAppName("SpeedCheck")
    val ssc = new StreamingContext(sparkConf, Seconds(1))
    val speedStream = ssc.socketTextStream("localhost", 9999)

    speedStream.map(checkSpeed)

    speedStream.print()

    ssc.start()
    ssc.awaitTermination()
  }

  def checkSpeed(speedValue: String): (String, Boolean) = {
    if (speedValue.asInstanceOf[Double] > SPEED_UPPER_LIMIT || speedValue.asInstanceOf[Double] < SPEED_LOWER_LIMIT) {
      (speedValue, false)
    } else {
      (speedValue, true)
    }
  }
}
\end{lstlisting}
\clearpage

\begin{lstlisting}[language=python,caption=edu.monash.honours.SpeedCheck (Python)]
__author__ = 'poltak'

from pyspark import SparkContext
from pyspark.streaming import StreamingContext

SPEED_UPPER_LIMIT = 85
SPEED_LOWER_LIMIT = 0
SOCKET_ADDRESS = "localhost"
SOCKET_PORT = 9999


def init_spark(name):
    sc = SparkContext("local[2]", name)
    return StreamingContext(sc, 2)


def close_spark(ssc):
    ssc.start()
    ssc.awaitTermination()


def check_speed(speed):
    try:
        if float(speed) > SPEED_UPPER_LIMIT or float(speed) < SPEED_LOWER_LIMIT:
            return speed, False
        else:
            return speed, True
    except ValueError:
        return speed, False


def main():
    ssc = init_spark("pyspark-speed-check")
    speed_stream = ssc.socketTextStream(SOCKET_ADDRESS, SOCKET_PORT)

    speed_stream = speed_stream.map(check_speed)

    speed_stream.pprint()

    close_spark(ssc)


if __name__ == '__main__':
    main()
\end{lstlisting}
