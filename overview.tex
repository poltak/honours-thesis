%!TEX root = thesis.tex
\section{DSPS Technology Overview}
\label{sec:overview}



\subsection{DSPS Technology Choices} % (fold)
\label{sub:dsps_technology_choices}

The choices to be focused on in this sub-project include the following DSPS technologies:

\begin{itemize}
  \item Samza
  \item Storm
  \item Spark Streaming
\end{itemize}

Literature concerning these DSPS technologies have been covered in~\sectref{sub:realtime_data_processing}, however
we will look into more depth into the systems regarding their interfaces.

% subsection dsps_technology_choices (end)



\subsection{Evaluation Method \& Approach} % (fold)
\label{sub:evaluation_method_approach}

Evaluation will be performed using both qualitative and quantitative methods. The quantitative evaluation methods used will focus
on the benchmarking of various features that are common to each of the technologies, and the comparison of the features that
each technology supports. The qualitative evaluation methods will focus on looking at the differences in ease-of-use,
support for different programming languages and features, and complexity of code written to implement the pipeline.

With looking at the decision of giving a clear recommendation for a particular technology out of the ones we have chosen,
we think it is import to look at both qualitative and quantitative aspects for comparison. These systems are significantly
non-trivial and vastly different in design and usage, however, as they still afford the same possible functionality,
it is very possible to give a properly constructed evaluation of them.


\subsubsection{Quantitative methods} % (fold)
\label{ssub:quantitative_methods}



% subsubsection quantitative_methods (end)


\subsubsection{Qualitative methods} % (fold)
\label{ssub:qualitative_methods}

The qualitative methods of evaluation will include the following:

% subsubsection qualitative_methods (end)

% subsection evaluation_method_approach (end)



\subsection{Overview of the data} % (fold)
\label{sub:overview_of_the_data}

As this sub-project focuses on the realtime processing of streaming data, data streams will have to be simulated from
datasets acquired from the Monash IRT team. An initial dataset has been given that includes the following layout of data:

%TODO: table from my other paper

The sample data acquired from the IRT team includes 99\,999 rows of data recorded from each of the mentioned sensors,
organised in a CSV file with headers. This is a general example of how data is currently received and handled by the IRT
team, however in the case of realtime data processing, data would be received in quite a different manner. Rather than
being received in large batches of readings, such as the sample dataset acquired, streams of data may be created from
each particular sensor, delivering data values to the DSPS systems a single value at a time along those streams. Data
is streamed in an asynchronous fashion, and can be simply thought of as DSPS system listens on an incoming stream, and
acts upon any data value whenever they may be received.

Hence, to simulate this from the sample we have acquired, the DSPS pipeline is constructed to listen on a particular
socket

% subsection overview_of_the_data (end)
