\begin{table}[h]
\caption{The different types of sensor data received from the Monash University Institute of Railway Technology team.}
\label{tab:irt_data}
\tabcolsep=0.11cm
\scalebox{0.7}{
\begin{tabular}{ | l | l | l | }

\hline
\textbf{Name}           & \textbf{Data type}                     & \textbf{Description} \\ \hline
\textbf{Time}           & Floating point value                   & Time in seconds since the start of the test. \\ \hline
\textbf{Speed}          & Floating point value                   & Speed in kilometres per hour. \\ \hline
\textbf{LoadEmpty}      & -1, 0, 1                               & 1 if load is present, 0 if empty, or -1 if unknown/in-port.  \\ \hline
\textbf{km}             & Floating point value                   & Elapsed kilometres travelled since the start of the test.\\ \hline
\textbf{Lat}            & Floating point value from -90 to +90   & Geographic latitude coordinate at given reading.  \\ \hline
\textbf{Lon}            & Floating point value from -180 to +180 & Geographic longitude coordinate at given reading.  \\ \hline
\textbf{Track}          & Integer value from 1 to 213\,841         & Given track in which the train is located.   \\ \hline
\textbf{SND1}           & Floating point value                   & Spring nest deflection count in millimetres at car corner.  \\ \hline
\textbf{SND2}           & Floating point value                   & Spring nest deflection count in millimetres at car corner.  \\ \hline
\textbf{SND3}           & Floating point value                   & Spring nest deflection count in millimetres at car corner.  \\ \hline
\textbf{SND4}           & Floating point value                   & Spring nest deflection count in millimetres at car corner.  \\ \hline
\textbf{CouplerForce}   & Floating point value                   & The amount of force couple detected in the train car.  \\ \hline
\textbf{LateralAccel}   & Floating point value                   & The amount of lateral acceleration detected in the train car.  \\ \hline
\textbf{AccLeft}        & Floating point value                   & The amount of acceleration detected by the left sensor.  \\ \hline
\textbf{AccRight}       & Floating point value                   & The amount of acceleration detected by the right sensor.  \\ \hline
\textbf{BounceFront}    & Floating point value                   & The amount of bounce detected at the front of the car.  \\ \hline
\textbf{BounceRear}     & Floating point value                   & The amount of bounce detected at the rear of the car.  \\ \hline
\textbf{RockFront}      & Floating point value                   & The amount of rock detected at the front of the car.  \\ \hline
\textbf{RockRear}       & Floating point value                   & The amount of rock detected at the front of the car.  \\ \hline

\end{tabular}}
\end{table}