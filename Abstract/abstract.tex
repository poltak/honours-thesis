% ************************** Thesis Abstract *****************************
% Use `abstract' as an option in the document class to print only the titlepage and the abstract.
\begin{abstract}

Data processing, using big data technologies, is quickly becoming more commonplace in data-focused applications. In the Monash
University Institute of Railway Technology, a project is being attempted which focuses on an existing railway project in
the Pilbara region of Western Australia. This project is looking at the integration of big data processing technologies for
use in processing readings from sensors monitoring the railway track conditions. Given planned future improvements in cellular
networks in the Pilbara, the project team are also considering the use of realtime data steam processing technologies for
the pre-processing, filtering, and live processing of the sensor readings acquired from the railways. These readings can
be used to detect and monitor both track conditions and train car conditions, allowing abnormalities to be detected.

As many different realtime data stream processing technologies are becoming more mature and readily available, this study
looks at forming a recommendation of the most suitable realtime data stream processing technology for use in the Monash
University Institute of Railway Technology project. Candidate technologies include Storm, Samza, and Spark Streaming. To
form this technology recommendation, a prototype pipeline will be designed and implemented in each of the
technologies, performing similar processing methods that are intended to be used in the Institute of Railway Technology
project.

Through the testing and evaluation of each candidate technology, using both quantitative and qualitative metrics, the study found that Storm is the most suitable technology
for the case-study at hand. Storm has shown to exhibit positive performance, and processing features that are of great interest
to the project.

The implications of this recommendation is that Storm will likely
be used to implement a realtime data filtering pipeline for use in the overall railway project. Furthermore, the prototype
pipeline, designed and implemented as part of this study, will be used as a baseline for the pipeline to be used in production
in the project.

Further research is recommended to assess the scalability of Storm on differently sized and configured clusters.
Furthermore, future research relating to Storm's ability at
handling batch data processing would also be of much interest to the Monash University
Institute of Railway Technology.

\end{abstract}
