%!TEX root = thesis.tex
\chapter{Conclusions}
\label{sec:conclusion}

This final chapter concludes this thesis by looking at a summary of the contributions made from our research in this
project, along with identifying possible areas of interest for any future work relating to this project. As this project
has a large overall scope, and the research discussed in this thesis only relates to a small part, many further opportunities
will present themselves for future work with the overall Monash University Institute of Railway Technology project.

This chapter begins with a summary of our research contribution, in~\sectref{sub:research_contribution}, before identifying
possible future work, in~\sectref{sub:future_work}.

\section{Research Contribution} % (fold)
\label{sub:research_contribution}

In this research, we gave a recommendation of a data stream processing system (DSPS) for realtime data processing use in the
Monash University Institute of Railway Technology (IRT) project. This project focuses around railway track condition monitoring
through the use of sensors placed on train cars. At inception of this research, the project had no realtime data processing
functionality integrated with it, however the engineering team working on the project expressed interest in using realtime
data processing for use with numerous functions required in their use-case; mostly pre-processing of sensor readings. Hence,
the aim of recommending an appropriate DSPS for use in the IRT project is to address these usage requirements, allowing
for such realtime processing to be used.

In the case of the IRT project, the DSPS technologies act as a pipeline through which data flows, all the way from the
sensor source to the destination. The destination is not specified, in the case of the IRT project, hence is left abstract for later extension. Suitable
destinations for data processed in realtime are data stores, such as a database management system or distributed file system,
or further processing, in the way of another DSPS pipeline or even batch processing systems. This pipeline requires design,
which often differs between different DSPS technologies, to adhere to the processing requirements of the use-case, and
consequently implementation.

In this research project, a very basic prototype pipeline has been designed and implemented to filter speed sensor readings.
This prototype pipeline has been designed in such a way that it is extensible and re-usable for different use-cases in the
IRT project. This pipeline has been implemented for use of evaluating different candidate DSPS technologies, however it is
also possible to use as a base-line for the design and implementation of the real pipeline to be used in production in the
IRT project.

This prototype pipeline implemented in this research has been used for testing and consequently evaluation of the different
candidate DSPS technologies, allowing us to arrive at the recommendation of the Storm DSPS for use in the IRT project.
Storm shows great overall performance in processing data in realtime, and also allows for extensible projects to be created,
suitable for large projects, intended to be in production for long periods of time, and subject to change. Furthermore,
Storm affords a great number of possible realtime and batch processing features, while exhibiting a relatively low
footprint on system resources.

Apart from the IRT project, which Storm is now intended to be used in production for, Storm has been in use for a number of years
for a range of different applications in numerous companies such as Twitter, Yahoo!, Baidu, and Spotify~\cite{storm_users}.

In summary, in relation to our original research questions, this research has answered all questions and met the aims of
the project. We have identified existing DSPS technologies for use in the IRT project through a review of existing literature
in the area of realtime data processing. These technologies include Storm, Samza, Spark Streaming, and S4. We have identified
a number of qualitative and quantitative metrics to evaluate each of the technologies based on the implementation of a
prototype realtime data processing pipeline in each of the technologies. For implementation, the candidate technologies
were narrowed to focus on Storm, Samza, and Spark Streaming due to the finding of S4 being effectively an abandoned project
in our review of existing literature. Finally, the prototype realtime data processing pipeline that was implemented was
designed to be extensible, allowing the addition of further realtime processing logic to be implemented as changes to the
IRT project become apparent.

% subsection research_contribution (end)


\section{Future Work} % (fold)
\label{sub:future_work}

During this research, potential areas for future work and research have been identified. These focus around looking at
the batch processing of data on the candidate DSPS technologies that support it, the evaluation of different DSPS technologies on different
cluster set-ups, and the evaluation of the use of the prototype pipeline in the overall IRT project. We discus these briefly.

Given that this research was focused solely around the realtime processing features of each of the candidate DSPS technologies,
it would be of further interest to look at both Spark Streaming and Storm's handling of batch data processing. As noted
in the evaluation chapter, both Spark Streaming, through the Spark batch processing engine, and Storm, through its
Trident extension, support batch processing performed on streaming data. As batch processing remains an important task
in many data-intensive applications, it would be of interest to evaluate these technologies based upon this factor, and also in relation
to existing batch processing systems, such as those that are categorised under the Hadoop ecosystem. Of course, this would
be of interest to users of a different processing use-case to that which was focused on in this research, hence would
not be appropriate for making a \textit{realtime} processing DSPS technology recommendation.

In relation to realtime data processing, further work of interest, building upon our evaluation of DSPS technologies, would
be to evaluate each of the DSPS processing technologies running across differently configured clusters of different node size.
The research performed in this project revolved around a single cluster set-up, being a single node cluster, used as a constant factor between all
tests. Looking at further cluster set-ups, this would allow an informed evaluation on how each technology scales in different environments.
Of interest would be the differences in performance, and the ease of distributed and parallel programming, afforded by the different DSPS technologies across
different cluster set-ups.

Lastly, it would be of interest to evaluate the prototype pipeline designed and implemented in this research, running on our recommended
DSPS technology, Storm, while in use with the entire IRT project ecosystem. Hence, it would be needed to be put in place,
between the sensors located on train cars, and the data storage layer, allowing the performance of it to be evaluated.
It may also be of interest to evaluate the recommended Storm pipeline against those pipelines implemented in the other candidate DSPS technologies,
that were also a result of this research, in relation to their use in place in the IRT project.

% subsection future_work (end)
